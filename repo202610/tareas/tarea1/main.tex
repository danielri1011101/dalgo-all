\usepackage[utf8]{inputenc}
\usepackage{amsmath}
\usepackage{listings}

% Number sets
\newcommand{\NN}{\mathbf{N}} % Natural numbers
\newcommand{\ZZ}{\mathbf{Z}} % Integers
\newcommand{\Zpos}{\ZZ^{+}} % Positive integers
\newcommand{\RR}{\mathbf{R}} % Real numbers
\newcommand{\Rpos}{\RR^{+}} % Positive reals

% Statement environments
\theoremstyle{definition}
% % Exercises
\newtheorem{ejer}{Ejercicio} 
\newtheorem*{ejer*}{Ejercicio} % No-number
% % Examples
\newtheorem{ejem}{Ejemplo}
\newtheorem*{ejem*}{Ejemplo} % No-number
% % Definitions
\newtheorem{defn}{Definición}


\begin{document}
\maketitle

\section{Matrices de Monge} Una matriz
\(
	A \in \mathbb{Z}^{m\times n}
\)
es una \emph{matriz de Monge} si dados índices $i, j, k, l$ tales que
\[
	\begin{array}{c}
		0 \leq i < k \leq m-1\\
		0 \leq j < l \leq n-1 
	\end{array}
\]

se cumple que
\[
	A^i_j + A^k_l \leq A^i_l + A^k_j.
\]

\textbf{Ejemplo.} La siguiente es una matriz de Monge $7\times 5$.
\[
	\left[
		\begin{array}{ccccc}
			10 & 17 & 13 & 28 & 23\\
			17 & 22 & 16 & 29 & 23\\
			24 & 28 & 22 & 34 & 24\\
			11 & 13 & 6 & 17 & 7\\
			45 & 44 & 32 & 37 & 23\\
			36 & 33 & 19 & 21 & 6\\
			75 & 66 & 51 & 53 & 34
		\end{array}
	\right]
\]

\subsection*{Teorema} Utilice la siguiente propiedad de las matrices
de Monge para resolver este ejercicio:
\begin{quote}
	\emph{La matriz $A$ es de Monge si y sólo si para cualesquiera índices $i,j$
		se cumple que}
	\[
		A^i_j + A^{i+1}_{j+1} \leq A^{i}_{j+1} + A^{i+1}_j.
	\]
\end{quote}

\subsection{Algoritmo iterativo} Escriba un algoritmo iterativo para determinar si
una matriz de números enteros es de Monge o no. Determine su complejidad temporal.

\subsection{Algoritmo de dividir y conquistar} Escriba un algoritmo de dividir y
conquistar para determinar si una matriz de números enteros es de Monge o no.
Plantee una ecuación de recurrencia que describa su complejidad temporal asumiendo
que se trata de una matriz $n\times n$ donde $n$ es una potencia de 2.

% nextsection
\section{Elemento mayoritario} Si \texttt{A} es un arreglo de elementos comparables
(números, caracteres, strings, \ldots) decimos que contiene un
\emph{elemento mayoritario} \texttt{a} si dicho elemento ocurre más de $n/2$ veces
donde $n$ es la longitud de \texttt{A}.

\subsection{Algoritmo} Escriba un algoritmo de dividir y conquistar que determine
si un arreglo contiene o no un elemento mayoritario. Aunque puede asumir que se
trata de un arreglo de números enteros, sólamente puede utilizar el operador
\texttt{==} para comparar elementos.

\subsection{Complejidad} Plantee una ecuación de recurrencia que determine la
complejidad temporal de su solución.

% nextsection
\section{Multiplicación rápida de enteros}
Para este problema consideramos que el tamaño de un entero $N$ es la longitud de su
expansión decimal, y esto lo denotamos con la letra minúscula correspondiente:
\[
	n= \textsf{size}(N)= \lfloor\log_{10} N\rfloor + 1.
\]

Por ejemplo
\[
	\begin{array}{l}
		N= 0 \Rightarrow n= 1\\
		N= 7 \Rightarrow n= 1\\
		N= 10 \Rightarrow n= 2\\
		N= 43 \Rightarrow n= 2\\
		N= 4310 \Rightarrow n= 4
	\end{array}
\]
Excluiremos los enteros negativos en este ejercicio. Es decir, si ejecutamos
\texttt{fastMult(m,n)} podemos asumir que $\mathtt{m}, \mathtt{n} \geq 0$.

\subsection*{Descripción del problema} Suponga que los enteros $N, M$ tienen la misma
longitud $n$. Entonces
\[
	NM = (10^{\lceil n/2 \rceil}X + Y)(10^{\lceil n/2 \rceil}W + Z)
\]
donde
\[
	\left\{
		\begin{array}{c}
			x = w = \mkfloor{n/2}\\
			y, z\leq \mkceil{n/2} 
		\end{array}
	\right.
\]

\bigskip
Desarrollando la expresión obtenemos que
\[
	NM =
	10^{2\mkceil{n/2}}XW + 10^{\mkceil{n/2}}(XZ + YW) + YZ.
\]

Entonces a simple vista calcular la multiplicación $NM$ nos requiere calcular cuatro
multiplicaciones de tamaño $\approx n/2$:
\[
	XW,\ XZ,\ YW,\ YZ.
\]
Dado que pensamos a un entero como la lista de sus dígitos, no nos preocupamos por
la complejidad de multiplicar por potencias de 10, ya que esto lo pensamos como
simples \emph{shifts} o corrimientos a la izquierda, agregando ceros al final de la
lista.

\subsection*{Idea del algoritmo} En vez de cuatro multiplicaciones, podemos usar las
siguientes tres:
\[
	P= XW,\ Q= YZ,\ R= (X+Y)(W+Z).
\]

De este modo obtenemos

\[
	NM =
	10^{2\mkceil{n/2}}P + 10^{\mkceil{n/2}}(R - P - Q) + Q.
\]
Recuerde que las operaciones de suma, resta y multiplicar por potencias de 10 se
consideran de complejidad despreciable con respecto a la de multiplicar.

\subsection{Estructura de datos} Implemente los enteros como listas enlazadas de
dígitos 0, 1, \ldots, 9. Puede usar listas \emph{doblemente enlazadas} si lo desea,
es decir puede guardar la información tanto de la cabeza como de la cola de la
lista junto con apuntadores de sucesor y predecesor. Defina las operaciones de suma,
resta y \emph{shift} sobre ellos. Para evitar usar números negativos, puede usar
una \emph{resta truncada}:
\[
	\texttt{subtract(m,n)}=
	\left\{
		\begin{array}{cc}
			\texttt{m-n} & \mbox{si \texttt{m} $\geq$ \texttt{n}}\\
			0 & \mbox{de otro modo.}
		\end{array}
	\right.
\]

\subsection{Algoritmo sin mejorar} Implemente mediante dividir y conquistar el
algoritmo que usa cuatro multiplicaciones y plantee una ecuación de recurrencia
que describa la complejidad temporal de este. Puede asumir que la complejidad de
shift, suma y resta es $O(n)$ donde $n$ es la longitud del número.

\subsection{Algoritmo rápido} Implemente mediante dividir y conquistar el
algoritmo mejorado que usa \emph{tres} multiplicaciones y plantee una ecuación de
recurrencia que describa la complejidad temporal de este. Puede asumir que la
complejidad de shift, suma y resta es $O(n)$ donde $n$ es la longitud del número.

\end{document}
