\usepackage[utf8]{inputenc}
\usepackage{amsmath}
\usepackage{listings}

% Number sets
\newcommand{\NN}{\mathbf{N}} % Natural numbers
\newcommand{\ZZ}{\mathbf{Z}} % Integers
\newcommand{\Zpos}{\ZZ^{+}} % Positive integers
\newcommand{\RR}{\mathbf{R}} % Real numbers
\newcommand{\Rpos}{\RR^{+}} % Positive reals

% Statement environments
\theoremstyle{definition}
% % Exercises
\newtheorem{ejer}{Ejercicio} 
\newtheorem*{ejer*}{Ejercicio} % No-number
% % Examples
\newtheorem{ejem}{Ejemplo}
\newtheorem*{ejem*}{Ejemplo} % No-number
% % Definitions
\newtheorem{defn}{Definición}


\begin{document}

\frame{\titlepage}

%

\begin{frame}{Información básica}

\begin{itemize}
    \item \textbf{Instructor:} Daniel Barrero $<$dr.barrero2562$>$ \pause
    \begin{itemize}
        \item \textbf{Horas de atención:} miércoles o viernes 11 AM (agendar previamente por e-mail).
        \item \textbf{Lugar de atención:} ML-763
    \end{itemize}
    \item[ ]\pause
    \item \textbf{Monitor:} Yefran Céspedes $<$y.cespedes$>$.
\end{itemize}
    
\end{frame}

%

\begin{frame}{Información básica}

\textbf{Calificación del curso} \pause

\begin{itemize}
    \item Parcial 1 [25\%]
    \item Parcial 2 [25\%]
    \item Parcial 3 [25\%] \pause
    \item Proyecto (3 entregas) [15\%] \pause
    \item Tareas y quices [10\%]
\end{itemize}

Los proyectos se desarrollan en parejas, las demás actividades son individuales. \\

\bigskip

\textbf{Política de aproximación de notas} \pause

\begin{itemize}
    \item Para aprobar el curso se debe obtener una nota sin aproximar de 3.0 o superior. \pause
    \item La mejor nota del curso será aproximada a 5.0 \pause
    \item No se hace aproximación de las demás notas finales.
\end{itemize}
    
\end{frame}

%

\begin{frame}{Bibliografía}
\begin{itemize}
    \item Cormen et al. \textit{Introduction to algorithms}. MIT Press, 2009.
    \item Bohórquez, Cardoso. \textit{Análisis de algoritmos}. Universidad de los Andes, 1992.
\end{itemize}\pause

\bigskip

\textbf{Otras referencias}

\begin{itemize}
    \item Brassard, Bratley. \textit{Fundamentals of algorithmics}. Prentice-Hall, 1996.
    \item Kocay, Kreher. \textit{Graphs, Algorithms, and Optimization}. CRC Press, 2017.
    \item Bellman, Dreyfus. \textit{Applied Dynamic Programming}. Princeton University Press, 1962.
    \item Aho, Hopcroft, Ullman. \textit{The Design and Analysis of Computer Algorithms}. Addison-Wesley, 1974.
\end{itemize}
\end{frame}

% nextframe
\begin{frame}{Why algorithmics? - Determinant example}
	Consider the task of computing the determinant of a square matrix.
	\begin{exercise}
		Compute
		\(
			\mathrm{det}(A)
		\)
		where
		\[
			A=
			\left(
				\begin{array}{ccc}
					3 & 1 & -2\\
					2 & 2 & 2\\
					5 & 0 & -1
				\end{array}
			\right)
		\]
	\end{exercise}
\end{frame}

% nextframe
\begin{frame}{Why algorithmics? - Determinant example}
	\textbf{Solution 1.} Apply the definition.

	\[
		\left|
		\begin{array}{ccc}
			3 & 1 & -2\\
			2 & 2 & 2\\
			5 & 0 & -1
		\end{array}
		\right| =
		3
		\left|
		\begin{array}{cc}
			2 & 2\\
			0 & -1
		\end{array}
		\right| -
		1
		\left|
		\begin{array}{cc}
			2 & 2\\
			5 & -1
		\end{array}
		\right| +
		(-2)
		\left|
		\begin{array}{cc}
			2 & 2\\
			5 & -1
		\end{array}
		\right|
	\]
	\[
		= 26
	\]
\end{frame}

% nextframe
\begin{frame}{Why algorithmics? - Determinant example}
	\textbf{Solution 1.} Apply the definition.

	\lstinputlisting[language=Java, firstline=10, lastline=26]{Aula01.java}
\end{frame}

% nextframe
\begin{frame}{Why algorithmics? - Determinant example}
	\textbf{Solution 2.} Apply Gauss-Jordan elimination.

	\begin{theorem}
		If $A$ is a square matrix with non-zero diagonal entries, there
		exist a lower-triangular matrix $L$ and an upper-triangular matrix
		$U$ such that
		\[
			LA = U.
		\]
		Therefore,
		\[
			\mathrm{det}(A) = \mathrm{det}(U)/\mathrm{det}(L).
		\]
	\end{theorem}
\end{frame}

% nextframe
\begin{frame}{Why algorithmics? - Determinant example}
	\textbf{Solution 2.} Apply Gauss-Jordan elimination.
	\[
		\left(
			\begin{array}{ccc}
				1 & 0 & 0\\
				0 & 1 & 0\\
				0 & 0 & 1
			\end{array}
		\right.
		\left|
			\begin{array}{ccc}
				3 & 1 & -2\\
				2 & 2 & 2\\
				5 & 0 & -1
			\end{array}
		\right)
		%
		\mapsto
		%
		\left(
			\begin{array}{ccc}
				1 & 0 & 0\\
				-2 & 3 & 0\\
				0 & 0 & 1
			\end{array}
		\right.
		\left|
			\begin{array}{ccc}
				3 & 1 & -2\\
				0 & 4 & 10\\
				5 & 0 & -1
			\end{array}
		\right)
		%
		\mapsto
	\]
	% nextequationline
	\[
		\left(
			\begin{array}{ccc}
				1 & 0 & 0\\
				-2 & 3 & 0\\
				-5 & 0 & 3
			\end{array}
		\right.
		\left|
			\begin{array}{ccc}
				3 & 1 & -2\\
				0 & 4 & 10\\
				0 & -5 & 7
			\end{array}
		\right)
		%
		\mapsto
		%
		\left(
			\begin{array}{ccc}
				1 & 0 & 0\\
				-2 & 3 & 0\\
				-30 & 15 & 12
			\end{array}
		\right.
		\left|
			\begin{array}{ccc}
				3 & 1 & -2\\
				0 & 4 & 10\\
				0 & 0 & 78
			\end{array}
		\right)
	\]
	% nextequationline
	\[
		=
		\left(
		\right.
		L
		\left|
		U
		\right)
	\]
\end{frame}

% nextframe
\begin{frame}{Why algorithmics? - Determinant example}
	\textbf{Solution 2.} Apply Gauss-Jordan elimination.

	\lstinputlisting[language=Java, firstline=77, lastline=91]{Aula01.java}

	\textit{Continue on the next page\ldots}
\end{frame}

%

\begin{frame}{Why Algorithmics? - Determinant example}
	\lstinputlisting[language=Java, firstline=93, lastline=105]{Aula01.java}

	\textbf{WARNING!} This algorithm can go wrong (i.e. return a number that is
	not the determinant).
	\begin{itemize}
		\item Exercise 1. Find three matrices for which the above given \texttt{gjDet}
			fails to compute the determinant.
		\item Exercise 2. Fix \texttt{gjDet} so that it always works.
	\end{itemize}
\end{frame}

%

\begin{frame}{Why Algorithmics? - Determinant example}
	Solution 1 has $O(n!)$ time complexity, and Solution 2 $O(n^3)$. Which one
	is best?

	\begin{itemize}
		\item Solution 2 is best \emph{eventually}.
	\end{itemize}

	\begin{eqnarray*}
		(n!)_{n=1}^{7} = 1, 2, 6, 24, 120, 720, 5040\\
		(n^3)_{n=1}^{7} = 1, 8, 27, 64, 125, 216, 343
	\end{eqnarray*}
\end{frame}

% nextframe
\begin{frame}{Why algorithmics? - Sorting}
	Recall the \emph{insertion sort} algorithm:\pause
	\lstinputlisting[language=Java, firstline=222, lastline=239]{Aula01.java}
	Looking at the nested loops, we see that insertion sort is
	\(
		O(n^2)
	\)
	in the worst case.
\end{frame}

% nextframe
\begin{frame}{Why algorithmics? - Sorting}
	On the other hand, we also have the well-known
	\emph{mergesort} algorithm:\pause
	\lstinputlisting[language=Java, firstline=161, lastline=181]{Aula01.java}
\end{frame}

% nextframe
\begin{frame}{Why algorithmics? - Sorting}
	Its time complexity satisfies the recurrence relation
	\[
		\left\{
			\begin{array}{l}
				t(1) = 1\\
				t(n) = t(\lfloor n/2 \rfloor) + t(\lceil n/2 \rceil)
				+ cn,\ n > 1.
			\end{array}
		\right.
	\]
	This gives
	\[
		t(n) \in O(n\log n).
	\]

	\begin{example}
		For an array of size $n= 1000$, 
		\begin{itemize}
			\item Insertion sort would take $\approx 1000000$ time units
				in the worst case.
			\item Mergesort would take $\approx 3000$ time units.
		\end{itemize}
	\end{example}
\end{frame}

% nextframe
\begin{frame}{Divide and Conquer}
	\centering
	\includegraphics[width=0.3\textwidth]{strategist.png}

	\bigskip
	\begin{quote}
		\emph{Unity is strength. Divide your enemy, and conquer them.}
	\end{quote}
\end{frame}
\end{document}
