\documentclass{beamer}

% Packages

% Sets

% Formal statements


\title{ISIS1105}
\author{Daniel R. Barrero R.}
\date{\today}

\begin{document}
\frame{\titlepage}

%

\begin{frame}{Información básica}

\begin{itemize}
    \item \textbf{Instructor:} Daniel Barrero $<$dr.barrero2562$>$ \pause
    \begin{itemize}
        \item \textbf{Horas de atención:} Jueves o Viernes 11 AM (agendar previamente por e-mail).
        \item \textbf{Lugar de atención:} ML-761
    \end{itemize}
    \item[ ]\pause
    \item \textbf{Monitores:} Elkin Cuello $<$e.cuello$>$, Juan David Duarte $<$j.duartey$>$.
\end{itemize}
    
\end{frame}

%

\begin{frame}{Información básica}

\textbf{Calificación del curso} \pause

\begin{itemize}
    \item Parcial 1 [20\%]
    \item Parcial 2 [20\%]
    \item Parcial 3 [20\%] \pause
    \item Proyecto (3 entregas) [20\%] \pause
    \item Talleres y Quices [20\%] \pause
\end{itemize}

Los proyectos se desarrollan en parejas, las demás actividades son individuales. \\

\bigskip

\textbf{Política de aproximación de notas} \pause

\begin{itemize}
    \item Para aprobar el curso es indispensable lograr una nota sin aproximar de 3.0 o superior. \pause
    \item La mejor nota del curso será aproximada a 5.0 \pause
    \item No se hace aproximación de las demás notas finales. \pause
\end{itemize}
    
\end{frame}

%

\begin{frame}{Bibliografía}
\begin{itemize}
    \item Cormen et al. \textit{Introduction to algorithms}. MIT Press, 2009.
    \item Bohórquez, Cardoso. \textit{Análisis de algoritmos}. Universidad de los Andes, 1992. \pause
    \item Brassard, Bratley. \textit{Algorithmics: theory and practice}. Prentice-Hall, 1988. \pause \smiley\smiley
\end{itemize}\pause

\bigskip

\textbf{Otras referencias}

\begin{itemize}
    \item Kocay, Kreher. \textit{Graphs, Algorithms, and Optimization}. CRC Press, 2017.
    \item Maurer, Ralston. \textit{Discrete algorithmic mathematics}. CRC Press, 2005.
\end{itemize}
\end{frame}

%

\begin{frame}{Why algorithmics? - Determinant example}
	Consider the task of computing the determinant of a square matrix.

	\bigskip

	\textbf{Solution 1.} Apply the definition.

	\lstinputlisting[language=Java, firstline=10, lastline=26]{Aula01.java}
\end{frame}
\end{document}
