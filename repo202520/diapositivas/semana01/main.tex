\documentclass{beamer}

\usepackage[utf8]{inputenc}
\usepackage{amsmath}
\usepackage{listings}

% Number sets
\newcommand{\NN}{\mathbf{N}} % Natural numbers
\newcommand{\ZZ}{\mathbf{Z}} % Integers
\newcommand{\Zpos}{\ZZ^{+}} % Positive integers
\newcommand{\RR}{\mathbf{R}} % Real numbers
\newcommand{\Rpos}{\RR^{+}} % Positive reals

% Statement environments
\theoremstyle{definition}
% % Exercises
\newtheorem{ejer}{Ejercicio} 
\newtheorem*{ejer*}{Ejercicio} % No-number
% % Examples
\newtheorem{ejem}{Ejemplo}
\newtheorem*{ejem*}{Ejemplo} % No-number
% % Definitions
\newtheorem{defn}{Definición}


\title{ISIS1105}
\author{Daniel R. Barrero R.}
\date{\today}

\begin{document}

\frame{\titlepage}

%

\begin{frame}{Información básica}

\begin{itemize}
    \item \textbf{Instructor:} Daniel Barrero $<$dr.barrero2562$>$ \pause
    \begin{itemize}
        \item \textbf{Horas de atención:} Jueves o Viernes 11 AM (agendar previamente por e-mail).
        \item \textbf{Lugar de atención:} ML-761
    \end{itemize}
    \item[ ]\pause
    \item \textbf{Monitores:} Elkin Cuello $<$e.cuello$>$, Juan David Duarte $<$j.duartey$>$.
\end{itemize}
    
\end{frame}

%

\begin{frame}{Información básica}

\textbf{Calificación del curso} \pause

\begin{itemize}
    \item Parcial 1 [20\%]
    \item Parcial 2 [20\%]
    \item Parcial 3 [20\%] \pause
    \item Proyecto (3 entregas) [20\%] \pause
    \item Talleres y Quices [20\%] \pause
\end{itemize}

Los proyectos se desarrollan en parejas, las demás actividades son individuales. \\

\bigskip

\textbf{Política de aproximación de notas} \pause

\begin{itemize}
    \item Para aprobar el curso es indispensable lograr una nota sin aproximar de 3.0 o superior. \pause
    \item La mejor nota del curso será aproximada a 5.0 \pause
    \item No se hace aproximación de las demás notas finales. \pause
\end{itemize}
    
\end{frame}

%

\begin{frame}{Bibliografía}
\begin{itemize}
    \item Cormen et al. \textit{Introduction to algorithms}. MIT Press, 2009.
    \item Bohórquez, Cardoso. \textit{Análisis de algoritmos}. Universidad de los Andes, 1992. \pause
    \item Brassard, Bratley. \textit{Algorithmics: theory and practice}. Prentice-Hall, 1988. \pause \smiley\smiley
\end{itemize}\pause

\bigskip

\textbf{Otras referencias}

\begin{itemize}
    \item Kocay, Kreher. \textit{Graphs, Algorithms, and Optimization}. CRC Press, 2017.
    \item Maurer, Ralston. \textit{Discrete algorithmic mathematics}. CRC Press, 2005.
\end{itemize}
\end{frame}

%

\begin{frame}{Why algorithmics? - Determinant example}
	Consider the task of computing the determinant of a square matrix.

	\textbf{Solution 1.} Apply the definition.

	\lstinputlisting[language=Java, firstline=10, lastline=26]{Aula01.java}
\end{frame}

%

\begin{frame}{Why algorithmics? - Determinant example}
	\textbf{Solution 2.} Apply Gauss-Jordan elimination.

	\lstinputlisting[language=Java, firstline=77, lastline=91]{Aula01.java}

	\textit{Continue on the next page\ldots}
\end{frame}

%

\begin{frame}{Why Algorithmics? - Determinant example}
	\lstinputlisting[language=Java, firstline=93, lastline=105]{Aula01.java}

	\textbf{WARNING!} This algorithm can go wrong (i.e. return a number that is
	not the determinant).
	\begin{itemize}
		\item Exercise 1. Find three matrices for which the above given \texttt{gjDet}
			fails to compute the determinant.
		\item Exercise 2. Fix \texttt{gjDet} so that it always works.
	\end{itemize}
\end{frame}

%

\begin{frame}{Why Algorithmics? - Determinant example}
	Solution 1 has $O(n!)$ time complexity, and Solution 2 $O(n^3)$. Which one
	is best?

	\begin{itemize}
		\item Solution 2 is best \emph{eventually}.
	\end{itemize}

	\begin{eqnarray*}
		(n!)_{n=1}^{7} = 1, 2, 6, 24, 120, 720, 5040\\
		(n^3)_{n=1}^{7} = 1, 8, 27, 64, 125, 216, 343
	\end{eqnarray*}
\end{frame}

%

\begin{frame}{Big-O notation}
	Let $f : \NN \to \Rpos$
	\begin{equation*}
		O(f) = \{\tau : \NN \to \Rpos | \exists N \in \NN, C \in \Rpos : \ \forall n \geq N : \ \tau(n) \leq C f(n)\}
	\end{equation*}
\end{frame}
\end{document}
