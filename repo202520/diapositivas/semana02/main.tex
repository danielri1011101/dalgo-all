\documentclass{beamer}

\usepackage[utf8]{inputenc}
\usepackage{amsmath}
\usepackage{listings}

% Number sets
\newcommand{\NN}{\mathbf{N}} % Natural numbers
\newcommand{\ZZ}{\mathbf{Z}} % Integers
\newcommand{\Zpos}{\ZZ^{+}} % Positive integers
\newcommand{\RR}{\mathbf{R}} % Real numbers
\newcommand{\Rpos}{\RR^{+}} % Positive reals

% Statement environments
\theoremstyle{definition}
% % Exercises
\newtheorem{ejer}{Ejercicio} 
\newtheorem*{ejer*}{Ejercicio} % No-number
% % Examples
\newtheorem{ejem}{Ejemplo}
\newtheorem*{ejem*}{Ejemplo} % No-number
% % Definitions
\newtheorem{defn}{Definición}


\title{Diseño y análisis de algoritmos \\ ISIS1105}
\author{Daniel R. Barrero R.}
\institute{Universidad de los Andes}

\begin{document}
\frame{\titlepage}

%

\begin{frame}{Big-O notation}
	Let $f : \NN \to \Rpos$. We define

	\begin{equation*}
		O(f) = \{\tau : \NN \to \Rpos | \exists C \in \Rpos, N \in \NN:
		\ \forall n \geq N:\ \tau(n) \leq Cf(n)\}
	\end{equation*}\pause

	\textbf{True or false?}
	\begin{itemize}
		\item $10n^2 \in O(n^3)$\pause
		\item $10n^2 \in O(n^2)$
		\item $n^3 \in O(10n^2)$\pause
		\item $n|\sin(n\pi/2)| \in O(n)$\pause
		\item $n \in O(n^2|\sin(n\pi/2)|)$
	\end{itemize}
\end{frame}

%

\begin{frame}{Recursive algorithms and recurrence relations}
	Consider the recursive way to compute Fibonacci numbers:

	\lstinputlisting[language=Java, firstline=23, lastline=28]{Aula03.java}
	\pause

	If we overlook the cost of addition, and let $t(n)$ denote the execution
	time of algorithm \texttt{fibRec} for input $n \in \NN$, we may assume that
	$t(n)$ satisfies the recurrence

	\begin{displaymath}
		\begin{cases}
			t(0)= 1,\ t(1)= 1 \\
			t(n)= t(n-1) + t(n-2),\ n \geq 2
		\end{cases}
	\end{displaymath}
\end{frame}

%

\begin{frame}{Linear homogeneous recurrence relations}
	Can we find a \emph{closed formula} for $t(n)$?\pause

	\bigskip
	Namely, can we solve the recurrence

	\begin{displaymath}
		\begin{cases}
			t(0)= 1,\ t(1)= 1 \\
			t(n)= t(n-1) + t(n-2),\ n \geq 2
		\end{cases}
	\end{displaymath}

	so that the value $t(n)$ depends only on $n$ and not on previous values
	of $t$?
\end{frame}

%

\begin{frame}{Linear homogeneous recurrence relations}
	Consider the sequence of powers $(a_n)_{n=0}^\infty$ given by $a_n = (-3)^n$.
	\pause

	\bigskip
	Namely,
	\begin{equation*}
		(a_n)_{n=0}^\infty = 1, -3, 9, -27, \cdots
	\end{equation*}\pause

	It is a solution of the recurrence
	\begin{displaymath}
		\begin{cases}
			t(0)= 1\\
			t(n)= -3t(n-1),\ n \geq 1
		\end{cases}
	\end{displaymath}\pause 

	But not of the recurrence
	\begin{displaymath}
		\begin{cases}
			t(0)= 4\\
			t(n)= -3t(n-1),\ n \geq 1
		\end{cases}
	\end{displaymath}
\end{frame}

%

\begin{frame}{Linear homogeneous recurrence relations}
	\textbf{Exercise}. Solve the recurrence
	\begin{displaymath}
		\begin{cases}
			t(0)= 4\\
			t(n)= -3t(n-1),\ n \geq 1
		\end{cases}
	\end{displaymath}\pause

	Notice that $(-3)^n$ is also a solution of
	\begin{displaymath}
		\begin{cases}
			t(0)= 1,\ t(1)= -3\\
			t(n)= -t(n-1) + 6t(n-2)
		\end{cases}
	\end{displaymath}\pause

	But not a solution of
	\begin{displaymath}
		\begin{cases}
			t(0)= 1,\ t(1)= 2\\
			t(n)= -t(n-1) + 6t(n-2)
		\end{cases}
	\end{displaymath}
\end{frame}

%

\begin{frame}{Linear homogeneous recurrence relations}
	\textbf{Exercise}. Solve the recurrence
	\begin{displaymath}
		\begin{cases}
			t(0)= 1,\ t(1)= 2\\
			t(n)= -t(n-1) + 6t(n-2)
		\end{cases}
	\end{displaymath}\pause

	\textbf{Exercise}. Find the recurrence of which $\sqrt{3}(-3)^n - \sqrt{2}\, 2^n$
	is the solution.
\end{frame}
\end{document}
