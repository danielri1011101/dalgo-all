\documentclass{beamer}

% Packages

% Sets

% Formal statements


\title{Diseño y análisis de algoritmos \\ ISIS1105}
\author{Daniel R. Barrero R.}
\institute{Universidad de los Andes}
\date{\today}

\begin{document}
\frame{\titlepage}

%

\begin{frame}{Algorithmic time complexity - formalities}
	\begin{defn}
		Let $\hat{Y}\ \mathtt{alg}(x : \hat{X})$ \{ \ldots \}
		be an algorithm, where $\hat{X}$ and $\hat{Y}$ are types
		of a programming language. Also, let
		\begin{equation*}
			|\cdot| : \hat{X} \to \NN
		\end{equation*}
		be the \emph{size} function of the elements of the input type
		$\hat{X}$.

		Given $n \in \NN$, we define $\hat{X}_n$ as the elements of
		$\hat{X}$ of size $n$, namely

		\begin{equation*}
			\hat{X}_n = \{x \in \hat{X} :\ |x| = n\}
		\end{equation*}
	\end{defn}
\end{frame}

%

\begin{frame}{Algorithmic time complexity - formalities}
	\begin{defn}
		Let 
		\begin{equation*}
			\hat{\tau}_{\mathtt{alg}}: \hat{X} \to \Rpos
		\end{equation*}
		be the \emph{time-cost function} of the algorithm \texttt{alg}. Namely,
		$\hat{\tau}_\mathtt{alg}(x)$ is the execution time of the algorithm
		\texttt{alg} for the input $x$.

		We define the \emph{worst-case time complexity function} of the
		algorithm \texttt{alg} as

		\begin{eqnarray*}
			\tau_\mathtt{alg} : \NN \to \Rpos \\
			\text{given by} \\
			\tau_\mathtt{alg}(n) = \max \ \{\hat{\tau}_\mathtt{alg} (x) :
			x \in \hat{X}_n\}
		\end{eqnarray*}
	\end{defn}
\end{frame}

%

\begin{frame}{Algorithmic time complexity - formalities}
	Therefore, when we say something like ``The algorithm \texttt{alg} is $O(n^2)$ ''
	what we actually mean is that $\tau_\mathtt{alg} \in O(n^2)$.
\end{frame}
\end{document}
