\usepackage[utf8]{inputenc}
\usepackage{amsmath}
\usepackage{listings}

% Number sets
\newcommand{\NN}{\mathbf{N}} % Natural numbers
\newcommand{\ZZ}{\mathbf{Z}} % Integers
\newcommand{\Zpos}{\ZZ^{+}} % Positive integers
\newcommand{\RR}{\mathbf{R}} % Real numbers
\newcommand{\Rpos}{\RR^{+}} % Positive reals

% Statement environments
\theoremstyle{definition}
% % Exercises
\newtheorem{ejer}{Ejercicio} 
\newtheorem*{ejer*}{Ejercicio} % No-number
% % Examples
\newtheorem{ejem}{Ejemplo}
\newtheorem*{ejem*}{Ejemplo} % No-number
% % Definitions
\newtheorem{defn}{Definición}


\begin{document}
\frame{\titlepage}

\begin{frame}{Computational complexity}
	\includegraphics[width= 0.6\textwidth]{mario.png}

	\bigskip

	\pause
	We've used notation such as $O$ and $\Omega$ to describe the worst-case complexity of
	\emph{algorithms}. We must now turn to the complexity of \emph{computational problems}. 
\end{frame}

%

\begin{frame}{Computational complexity --- \emph{Church-Turing thesis and models of computation}}
	\textbf{Models of computation}
	\begin{itemize}
		\item Turing machines \checkmark (ISIS1106)
		\item RAM instructions \checkmark (Every other course)
		\item Lambda calculus
		\item Arithmetic circuits
		\item Boolean circuits
	\end{itemize}

	\bigskip

	\pause
	The \textbf{Church-Turing thesis} says that any two of these models are equivalent.

	\bigskip

	\pause
	Our discussion of computational complexity and so-called ``intractability'' will be in
	terms of the \emph{RAM} model of computation. 
\end{frame}

%

\begin{frame}{RAM instructions model of computation}
	\begin{itemize}
		\item We will use a \textbf{fixed programming language \underline{with types}},
			very Java-like.
		\item[] {\scriptsize \it *We may nonetheless occasionally discuss some statements
			in terms of Turing machines*}
	\end{itemize}
\end{frame}

%

\begin{frame}{Decision problems}
	\begin{defn}
		Let $\hat{X}$ be a type of our programming language and let
		$X \subset \hat{X}$. The \emph{decision problem} defined by $X$ consists
		of assigning \textbf{true} or \textbf{false} to the expression $x \in X$
		for all $x \in \hat{X}$. 
	\end{defn}

	\pause
	\begin{exl}
		\begin{enumerate}
			\item $\hat{X}= \ZZ$, $X=$ Primes.
			\item $\hat{X}= \mathtt{Graph}$, $X= \mathtt{connGraph}$. Namely,
				the problem of deciding whether or not a graph is
				connected.
			\item $\hat{X}= \mathtt{String}$, $X=$ Python programs that stop.
		\end{enumerate}
	\end{exl}

	\pause
	\textbf{Notation.} We will use $X \subset \hat{X}$ to denote the decision problem
	defined by $X$.
\end{frame}

%

\begin{frame}{Decision problems - computability}
	\begin{defn}
		The decision problem $X \subset \hat{X}$ is \emph{computable} if there
		exists a program
		\[
			\mathbf{boolean}\ \mathtt{checkX}(\hat{X}\ x)
		\]

		such that \(\mathtt{checkX}(x) =\) \textbf{true} if and only if
		$x \in X$.
	\end{defn}

	\pause
	\begin{exl}
		Examples 1 and 2 from the previous slide are computable, example 3 is
		not.
	\end{exl}
\end{frame}

%

\begin{frame}{Computable decision problems - Complexity classes}
	\begin{defn}
		The computable decision problem $X \subset \hat{X}$ is
		\emph{of class $P$} or is said to have \emph{polynomial time complexity}
		if there exists $k \in \Zpos$ such that
		\[
			\mathtt{checkX} \in O(n^k),
		\]
		where $n$ is the size of the input $x \in \hat{X}$.
	\end{defn}

	\pause
	\begin{exl}
		\begin{enumerate}
			\item The problem
				\(
					\text{Primes} \subset \ZZ \in P
				\)
				by taking \texttt{checkX} $=$ \texttt{isPrime} from the
				next slide.
			\item The problem \texttt{connGraph} $\subset$ \texttt{Graph}
		\end{enumerate}
	\end{exl}
\end{frame}

%

\begin{frame}{Computable decision problems - Complexity classes}
	\begin{defn}
		The computable decision problem $X \subset \hat{X}$ is
		\emph{of class $NP$} or is said to have
		\emph{non-deterministic polynomial time complexity} if there exist
		a type $Q$, a decision problem $Z \subset \hat{X} \times Q$, and a
		polynomial $p(n)$ satisfying the following conditions:
		\begin{itemize}
			\item $x \in X$ if and only if there exists $q \in Q$ such that
				$(x, q) \in Z$.
			\item For all $(x, q) \in Z$ the size of $q$ is at most $p(n)$
				where $n$ is the size of $x$.
			\item \texttt{checkZ} is of class $P$.
		\end{itemize}

		The decision problem $Z \subset \hat{X} \times Q$ is called a
		\emph{proof system} for $X \subset \hat{X}$.
	\end{defn}
\end{frame}

%

\begin{frame}{NP problems}
	\begin{exl}
		The problem
		\[
			\overline{\text{Primes}} \subset \ZZ
		\]
		of wheter or not a number is composite admits the proof system
		\[
			Z= \{(n, a) : 1 < |a| < |n|,\ \ n \ 
				\mathbf{mod}\ a = 0\}
		\]
		and
		\(
			Z \subset \ZZ \times \ZZ \in P
		\)
		since it consists of dividing $n$ by $a$.
	\end{exl}
\end{frame}
\end{document}
