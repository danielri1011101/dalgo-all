\documentclass{beamer}

\usepackage[utf8]{inputenc}
\usepackage{amsmath}
\usepackage{listings}

% Number sets
\newcommand{\NN}{\mathbf{N}} % Natural numbers
\newcommand{\ZZ}{\mathbf{Z}} % Integers
\newcommand{\Zpos}{\ZZ^{+}} % Positive integers
\newcommand{\RR}{\mathbf{R}} % Real numbers
\newcommand{\Rpos}{\RR^{+}} % Positive reals

% Statement environments
\theoremstyle{definition}
% % Exercises
\newtheorem{ejer}{Ejercicio} 
\newtheorem*{ejer*}{Ejercicio} % No-number
% % Examples
\newtheorem{ejem}{Ejemplo}
\newtheorem*{ejem*}{Ejemplo} % No-number
% % Definitions
\newtheorem{defn}{Definición}


\title{Diseño y análisis de algoritmos \\ ISIS1105}
\author{Daniel R. Barrero R.}
\institute{Universidad de los Andes}

\begin{document}
\frame{\titlepage}

%

\begin{frame}{LRRs - Homogenization trick}
	\begin{thm}[2]\label{ht}
		If the initial value problem
		\begin{displaymath}(4)
			\begin{cases}
				t(0)= v_0, \cdots, t(k-1)= v_{k-1}\\
				t(n) + \sum_{i= 1}^k b_it(n-i)= q,\ n \geq k
			\end{cases}
		\end{displaymath}
		is such that $q = c^np(n)$, where $c$ is a positive constant and
		$p(n)$ is a polynomial of degree $d$, then it is equivalent to a
		homogeneous problem of degree $k+d+1$ with characteristic equation
		\begin{equation*}
			E(x)(x-c)^{d+1} = 0
		\end{equation*}
		where $E(x) = 0$ is the characteristic equation of the recurrence
		in (4).
	\end{thm}
\end{frame}

%% Divide and conquer
\begin{frame}{Divide and conquer}
	\centering
	\includegraphics[width=0.3\textwidth]{strategist.png}
	
	\bigskip
	``\emph{Unity is strength. Divide your enemy, and conquer them.}''
\end{frame}

%

\begin{frame}{D \& C - fast exponentiation}
	Consider the following algorithm
	
	\lstinputlisting[language=Java, firstline=8, lastline=18]{Aula05.java}
	
	Let $t(n)$ denote the number of multiplications \texttt{'*'}. It defines the
	initial value problem
	\[
		\begin{cases}
			t(0)= 0\\
			t(n)= t(n/2) + 1,\ n > 0\ \text{even}\\
			t(n)= t(n-1) + 1,\ n > 0\ \text{odd}.
		\end{cases}
	\]
\end{frame}

%

\begin{frame}{D \& C - fast exponentiation}
	To compute its time complexity, notice that
	\[
		\lfloor n/2 \rfloor =
		\begin{cases}
			n/2,\ n \text{ even}\\
			(n-1)/2,\ n \text{ odd}.
		\end{cases}
	\]
	Therefore, if $n$ is odd
	\[
		\begin{array}{c}
			t(n) = \left( t\left( \frac{n-1}{2} \right) + 1 \right) + 1\\
			= t(\lfloor n/2 \rfloor) + 2.
		\end{array}
	\]
	Then $t(n) \in O(\tau(n))$ where
	\[
		\begin{cases}
			\tau(0)= 0\\
			\tau(n)= \tau(\lfloor n/2 \rfloor) + 2,\ n > 0.
		\end{cases}
	\]
\end{frame}

%

\begin{frame}{D \& C - fast exponentiation}
	We apply the change of variable $n= 2^m$ and let $T(m)= \tau(2^m)$. We
	obtain the i.v.p.
	\[
		\begin{cases}
			T(0)= 2\\
			T(m)= T(m-1) + 2,\ m > 0.
		\end{cases}
	\]
	We apply the homogenization trick. If $m > 1$ then
	\[
		\begin{array}{c}
			T(m) - T(m-1) = 2\\
			T(m-1) - T(m-2) = 2.
		\end{array}
	\]
	Then a closed formula for $T(m)$ is obtained by solving the homogeneous i.v.p.
	\[
		\begin{cases}
			T(0)= 2,\ T(1) = 4\\
			T(m) - 2T(m-1) + T(m-2)= 0,\ m > 1.
		\end{cases}
	\]
\end{frame}

%

\begin{frame}{D \& C - fast exponentiation}
	The characteristic equation is $(x - 1)^2= 0$, so the solution is
	\[
		T(m) = u1^m + vm1^m
	\]
	where the constants $u$ and $v$ are such that
	\[
		\begin{cases}
			u = 2\\
			u + v = 4.
		\end{cases}
	\]
	This gives $T(m) = 2 + 2m$. Since $m = \log n$, we obtain
	\[
		\tau(n) = 2 + 2\log n
	\]
	and we conclude $t(n) \in O(\log n)$.
\end{frame}

%

\begin{frame}{D \& C - master theorem}
	Let $t(n)$ be such that
	\[
		t(n) \leq at(n/b) + cn^k
	\]
	for $c \in \Rpos$, $a \in \Zpos$, $b \in \Zpos$, $k \in \NN$ and $n$ a
	large enough power of $b$. Then
	\[
		t(n) \in
		\begin{cases}
			O(n^k)\ \ \text{if}\ a < b^k\\
			O(n^k\log n)\ \ \text{if}\ a = b^k\\
			O(n^{\log_b a})\ \ \text{if}\ a > b^k.
		\end{cases}
	\]
\end{frame}
\end{document}
