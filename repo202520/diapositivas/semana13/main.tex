\usepackage[utf8]{inputenc}
\usepackage{amsmath}
\usepackage{listings}

% Number sets
\newcommand{\NN}{\mathbf{N}} % Natural numbers
\newcommand{\ZZ}{\mathbf{Z}} % Integers
\newcommand{\Zpos}{\ZZ^{+}} % Positive integers
\newcommand{\RR}{\mathbf{R}} % Real numbers
\newcommand{\Rpos}{\RR^{+}} % Positive reals

% Statement environments
\theoremstyle{definition}
% % Exercises
\newtheorem{ejer}{Ejercicio} 
\newtheorem*{ejer*}{Ejercicio} % No-number
% % Examples
\newtheorem{ejem}{Ejemplo}
\newtheorem*{ejem*}{Ejemplo} % No-number
% % Definitions
\newtheorem{defn}{Definición}


\begin{document}
\frame{\titlepage}

%

\begin{frame}{Reductions}
	\begin{dfn}
		We say that the problem
		\(
			X \subset \hat{X}
		\)
		is \emph{polynomial time reducible} to the problem
		\(
			Y \subset \hat{Y}
		\)
		if there exists an algorithm
		\[
			\hat{Y} \ \mathtt{translate}(\hat{X} \ x)
		\]
		such that
		\begin{enumerate}
			\item
				\(
					\tau_{\mathtt{translate}} \in O(n^k)
				\)
				for some $k \in \Zpos$ where $n= |x|$.
			\item
				\(
					X = \mathtt{translate}^{-1}(Y).
				\)	  
		\end{enumerate}

		We will denote this by
		\[
			X \subset \hat{X} \leq_P Y \subset \hat{Y}
		\]
		and simply say \emph{reducible} instead of ``polynomial time reducible''.
	\end{dfn}
\end{frame}

%

\begin{frame}{Theorem: CovD $\leq_P$ HamD}
	\begin{dfn}
		A \emph{widget} for an edge $\{u, v\}$ is the graph

		\bigskip
		\begin{tikzpicture}
			% uvi nodes
			\node[shape= circle, inner sep= 2pt, draw, label=above:uv1] (uv1) at (0, 0) {};
			\node[shape= circle, inner sep= 2pt, draw] (uv2) at (0, -1) {};
			\node[shape= circle, inner sep= 2pt, draw] (uv3) at (0, -2) {};
			\node[shape= circle, inner sep= 2pt, draw] (uv4) at (0, -3) {};
			\node[shape= circle, inner sep= 2pt, draw] (uv5) at (0, -4) {};
			\node[shape= circle, inner sep= 2pt, draw] (uv6) at (0, -5) {};
			% vui nodes
			\node[shape= circle, inner sep= 2pt, draw] (vu1) at (1, 0) {};
			\node[shape= circle, inner sep= 2pt, draw] (vu2) at (1, -1) {};
			\node[shape= circle, inner sep= 2pt, draw] (vu3) at (1, -2) {};
			\node[shape= circle, inner sep= 2pt, draw] (vu4) at (1, -3) {};
			\node[shape= circle, inner sep= 2pt, draw] (vu5) at (1, -4) {};
			\node[shape= circle, inner sep= 2pt, draw, label=below:vu6] (vu6) at (1, -5) {};
			% vertical uv connections
			\draw (uv1) -- (uv2);
			\draw (uv2) -- (uv3);
			\draw (uv3) -- (uv4);
			\draw (uv4) -- (uv5);
			\draw (uv5) -- (uv6);
			% vertical vu connections
			\draw (vu1) -- (vu2);
			\draw (vu2) -- (vu3);
			\draw (vu3) -- (vu4);
			\draw (vu4) -- (vu5);
			\draw (vu5) -- (vu6);
			% crossings
			\draw (uv1) -- (vu3);
			\draw (vu1) -- (uv3);
			\draw (uv4) -- (vu6);
			\draw (vu4) -- (uv6);
		\end{tikzpicture}
	\end{dfn}
\end{frame}

%

\begin{frame}{Theorem: CovD $\leq_P$ HamD}
		Widgets can be transversed in four ways:

		\bigskip
		\begin{tikzpicture}
			% uvi nodes
			\node[shape= circle, inner sep= 2pt, draw, label=above:uv1] (uv1) at (0, 0) {};
			\node[shape= circle, inner sep= 2pt, draw] (uv2) at (0, -1) {};
			\node[shape= circle, inner sep= 2pt, draw] (uv3) at (0, -2) {};
			\node[shape= circle, inner sep= 2pt, draw] (uv4) at (0, -3) {};
			\node[shape= circle, inner sep= 2pt, draw] (uv5) at (0, -4) {};
			\node[shape= circle, inner sep= 2pt, draw, label=below:uv6] (uv6) at (0, -5) {};
			% vui nodes
			\node[shape= circle, inner sep= 2pt, draw] (vu1) at (1, 0) {};
			\node[shape= circle, inner sep= 2pt, draw] (vu2) at (1, -1) {};
			\node[shape= circle, inner sep= 2pt, draw] (vu3) at (1, -2) {};
			\node[shape= circle, inner sep= 2pt, draw] (vu4) at (1, -3) {};
			\node[shape= circle, inner sep= 2pt, draw] (vu5) at (1, -4) {};
			\node[shape= circle, inner sep= 2pt, draw] (vu6) at (1, -5) {};
			% vertical uv connections
			\draw[color=blue,thick] (uv1) -- (uv2);
			\draw[color=blue,thick] (uv2) -- (uv3);
			\draw[color=blue,thick] (uv3) -- (uv4);
			\draw[color=blue,thick] (uv4) -- (uv5);
			\draw[color=blue,thick] (uv5) -- (uv6);
			% vertical vu connections
			\draw (vu1) -- (vu2);
			\draw (vu2) -- (vu3);
			\draw (vu3) -- (vu4);
			\draw (vu4) -- (vu5);
			\draw (vu5) -- (vu6);
			% crossings
			\draw (uv1) -- (vu3);
			\draw (vu1) -- (uv3);
			\draw (uv4) -- (vu6);
			\draw (vu4) -- (uv6);
		\end{tikzpicture}
\end{frame}

%

\begin{frame}{Theorem: CovD $\leq_P$ HamD}
		Widgets can be transversed in four ways:

		\bigskip
		\begin{tikzpicture}
			% uvi nodes
			\node[shape= circle, inner sep= 2pt, draw, label=above:uv1] (uv1) at (0, 0) {};
			\node[shape= circle, inner sep= 2pt, draw] (uv2) at (0, -1) {};
			\node[shape= circle, inner sep= 2pt, draw] (uv3) at (0, -2) {};
			\node[shape= circle, inner sep= 2pt, draw] (uv4) at (0, -3) {};
			\node[shape= circle, inner sep= 2pt, draw] (uv5) at (0, -4) {};
			\node[shape= circle, inner sep= 2pt, draw, label=below:uv6] (uv6) at (0, -5) {};
			% vui nodes
			\node[shape= circle, inner sep= 2pt, draw] (vu1) at (1, 0) {};
			\node[shape= circle, inner sep= 2pt, draw] (vu2) at (1, -1) {};
			\node[shape= circle, inner sep= 2pt, draw] (vu3) at (1, -2) {};
			\node[shape= circle, inner sep= 2pt, draw] (vu4) at (1, -3) {};
			\node[shape= circle, inner sep= 2pt, draw] (vu5) at (1, -4) {};
			\node[shape= circle, inner sep= 2pt, draw] (vu6) at (1, -5) {};
			% vertical uv connections
			\draw[color=blue,thick] (uv1) -- (uv2);
			\draw[color=blue,thick] (uv2) -- (uv3);
			\draw (uv3) -- (uv4);
			\draw[color=blue,thick] (uv4) -- (uv5);
			\draw[color=blue,thick] (uv5) -- (uv6);
			% vertical vu connections
			\draw[color=blue,thick] (vu1) -- (vu2);
			\draw[color=blue,thick] (vu2) -- (vu3);
			\draw[color=blue,thick] (vu3) -- (vu4);
			\draw[color=blue,thick] (vu4) -- (vu5);
			\draw[color=blue,thick] (vu5) -- (vu6);
			% crossings
			\draw (uv1) -- (vu3);
			\draw[color=blue,thick] (vu1) -- (uv3);
			\draw[color=blue,thick] (uv4) -- (vu6);
			\draw (vu4) -- (uv6);
		\end{tikzpicture}
\end{frame}
\end{document}
