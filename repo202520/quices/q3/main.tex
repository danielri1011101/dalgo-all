\documentclass{amsart}

\usepackage[utf8]{inputenc}
\usepackage{amsmath}
\usepackage{listings}

% Number sets
\newcommand{\NN}{\mathbf{N}} % Natural numbers
\newcommand{\ZZ}{\mathbf{Z}} % Integers
\newcommand{\Zpos}{\ZZ^{+}} % Positive integers
\newcommand{\RR}{\mathbf{R}} % Real numbers
\newcommand{\Rpos}{\RR^{+}} % Positive reals

% Statement environments
\theoremstyle{definition}
% % Exercises
\newtheorem{ejer}{Ejercicio} 
\newtheorem*{ejer*}{Ejercicio} % No-number
% % Examples
\newtheorem{ejem}{Ejemplo}
\newtheorem*{ejem*}{Ejemplo} % No-number
% % Definitions
\newtheorem{defn}{Definición}


\title{ISIS1105 \\ Quiz 3}
\author{Daniel R. Barrero R.}

\begin{document}
\maketitle

\section{} ¿Cuáles de las siguientes funciones pertenecen a $O(n)$? Seleccione
todas las que apliquen.
\begin{tabbing}
	(I). $n + \log n$ \= (II). $n\sqrt{n}$\\
	(III). $\log n$   \> (IV). $n + \sqrt{n}$
\end{tabbing}
\begin{enumerate}
	\item[(a)] I, II y IV
	\item[(b)] I, III y IV
	\item[(c)] III
	\item[(d)] Todas pertenecen.
\end{enumerate}

%

\section{} Considere la recurrencia lineal no-homogénea
\[
	\begin{cases}
		t(0)= 1\\
		t(n)= 3t(n-1) + 2^n,\ n > 0.
	\end{cases}
\]
¿Cuál es el problema de valor inicial homogéneo que resulta de aplicarle el
\emph{homogenization trick}?
\begin{enumerate}
	\item[(a)]
		\[
			\begin{cases}
				t(0)= 1,\ t(1)= 2\\
				t(n)= -5t(n-1) + 6t(n-2),\ n > 1
			\end{cases}
		\]
	\item[(b)]
		\[
			\begin{cases}
				t(0)= 1,\ t(1)= 5\\
				t(n)= 5t(n-1) + 6t(n-2),\ n > 1
			\end{cases}
		\]
	\item[(c)]
		\[
			\begin{cases}
				t(0)= 1,\ t(1)= 5\\
				t(n)= 5t(n-1) - 6t(n-2),\ n > 1
			\end{cases}
		\]
	\item[(d)]
		\[
			\begin{cases}
				t(0)= 1,\ t(1)= 2\\
				t(n)= 5t(n-1) - 6t(n-2),\ n > 1
			\end{cases}
		\]
\end{enumerate}

%

\section{} Considere la recurrencia lineal no-homogénea
\[
	\begin{cases}
		t(0)= 0\\
		t(n)= t(n-1) + n,\ n > 0.
	\end{cases}
\]
¿Cuál es el problema de valor inicial homogéneo que resulta de aplicarle el
\emph{homogenization trick}?
\begin{enumerate}
	\item[(a)]
		\[
			\begin{cases}
				t(0)= 0,\ t(1)= 1,\ t(2)= 2\\
				t(n)= 3t(n-1) - 3t(n-2) + t(n-3),\ n > 2
			\end{cases}
		\]
	\item[(b)]
		\[
			\begin{cases}
				t(0)= 0,\ t(1)= 1,\ t(2)= 3\\
				t(n)= 3t(n-1) - 3t(n-2),\ n > 2
			\end{cases}
		\]
	\item[(c)]
		\[
			\begin{cases}
				t(0)= 0,\ t(1)= 1,\ t(2)= 3\\
				t(n)= 3t(n-1) - 3t(n-2) + t(n-3),\ n > 2
			\end{cases}
		\]
	\item[(d)]
		\[
			\begin{cases}
				t(0)= 0,\ t(1)= 1,\ t(2)= 3\\
				t(n)= t(n-1) - t(n-2) + t(n-3),\ n > 2
			\end{cases}
		\]
\end{enumerate}

%

\section{} Considere la recurrencia lineal no-homogénea donde $c$ y
$c'$ son constantes.
\[
	\begin{cases}
		t(0)= c,\ t(1)= c'\\
		t(n)= 3t(n-1) - 2t(n-2) + 1,\ n > 1.
	\end{cases}
\]
¿Qué forma tiene su solución?
\begin{enumerate}
	\item[(a)] $u_0 + u_1n + u_2n^2$
	\item[(b)] $u_0 + v_0 2^n + v_1n2^n$
	\item[(c)] $v_02^n + v_1n2^n + v_2n^22^n$
	\item[(d)] $u_0 + u_1n + v_02^n$
\end{enumerate}\end{document}
