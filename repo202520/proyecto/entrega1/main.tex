\documentclass{amsart}
% Packages

% Sets

% Formal statements


\title{Proyecto 1 \\ ISIS1105 202520}
\author{Rubén Manrique}

\begin{document}
\maketitle

\section*{0. Objetivos}

\section{Condiciones generales}

\section{Descripción del problema}	
En la ciudad futurista de Tecnotown se celebra anualmente el Festival de los
Robots Creativos. Uno de los desafíos consiste en que cada robot debe distribuir
una cantidad total de energía $n$ en exactamente $k$ celdas, de modo que la suma de
las energías asignadas a cada celda corresponda al total de energía $n$.

\bigskip
El jurado del festival considera creativas únicamente aquellas celdas cuyos números
contienen los dígitos 3, 6 o 9. La cantidad de puntos de creatividad obtenida por
una celda depende tanto del dígito como de su posición en el número, según la
siguiente tabla:
\end{document}
