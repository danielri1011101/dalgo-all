\usepackage[utf8]{inputenc}
\usepackage{amsmath}
\usepackage{listings}

% Number sets
\newcommand{\NN}{\mathbf{N}} % Natural numbers
\newcommand{\ZZ}{\mathbf{Z}} % Integers
\newcommand{\Zpos}{\ZZ^{+}} % Positive integers
\newcommand{\RR}{\mathbf{R}} % Real numbers
\newcommand{\Rpos}{\RR^{+}} % Positive reals

% Statement environments
\theoremstyle{definition}
% % Exercises
\newtheorem{ejer}{Ejercicio} 
\newtheorem*{ejer*}{Ejercicio} % No-number
% % Examples
\newtheorem{ejem}{Ejemplo}
\newtheorem*{ejem*}{Ejemplo} % No-number
% % Definitions
\newtheorem{defn}{Definición}


\begin{document}
\maketitle

\section{Descripción del problema}	
En la ciudad futurista de Tecnotown se celebra anualmente el Festival de los
Robots Creativos. Uno de los desafíos consiste en que cada robot debe distribuir
una cantidad total de energía $n \in \Zpos$ en exactamente $k \in \Zpos$ celdas, de
modo que la suma de las energías asignadas a cada celda corresponda al total de
energía $n$.

\subsection{Puntuación}
Una celda otorga puntos de acuerdo a los dígitos de la expansión decimal de la
cantidad de energía almacenada en ella. Sólamente los dígitos múltiplos de 3 otorgan
puntos, de acuerdo a su posición en la expansión decimal. Las únicas posiciones
relevantes son las primeras cinco, leídas de derecha a izquierda. La fórmula del
puntaje entregado por una celda es la siguiente:

\begin{equation}\label{cellscore}
	P= v_0P_0 + v_1 P_1 + v_2 P_2 + v_3 P_3 + v_4 P_4
\end{equation}

donde $P_i$ es el puntaje asignado por los jueces a la posición $i$-ésima y

\begin{equation}\label{vi}
	v_i= \alpha(d_i)
\end{equation}

donde

\begin{equation}\label{alpha}
	\alpha(m)=
	\left\{
		\begin{array}{ll}
			0, & \mbox{$m$ \textbf{mod} $3 \not= 0$}\\
			m/3, & \mbox{$m$ \textbf{mod} $3 = 0$}
		\end{array}
	\right.
\end{equation}

donde

\begin{equation}\label{edivalg}
	E = d_010^0 + d_110^1 + d_210^2 + d_310^3 + d_410^4 + q10^5
\end{equation}

según el algoritmo de la división y $E$ es la cantidad de energía almacenada en
la celda.

\bigskip
De modo que el robot ganador es aquel que encuentre la manera de distribuir la energía
en las celdas maximizando el puntaje.

%

\section{Ejercicios}
\begin{enumerate}
	\item Para cada uno de los casos de prueba del enunciado, escoja una distribución
		(no necesariamente óptima) de la energía en las celdas y calcule todos los valores
		---
		\(
			P_0, P_1, \ldots, P_4, d_i, v_i
		\)
		---que aparecen en las ecuaciones \eqref{cellscore}, \eqref{vi}, \eqref{alpha} y
		\eqref{edivalg}.
	\item Determine, para cada uno de los casos de prueba del enunciado, cuáles deberían ser
		las dimensiones de la tabla \texttt{memo}, la cual se va completando a lo largo de
		la ejecución del algoritmo.
	\item Compare este problema con los otros problemas de programación dinámica que se han
		resuelto en clase y/o tareas y determine cúal es el que usted encuentre más similar,
		justificando su respuesta en sus propias palabras.
\end{enumerate}
\end{document}
