\documentclass{amsart}
\usepackage[utf8]{inputenc}
\usepackage{amsmath}
\usepackage{listings}

% Number sets
\newcommand{\NN}{\mathbf{N}} % Natural numbers
\newcommand{\ZZ}{\mathbf{Z}} % Integers
\newcommand{\Zpos}{\ZZ^{+}} % Positive integers
\newcommand{\RR}{\mathbf{R}} % Real numbers
\newcommand{\Rpos}{\RR^{+}} % Positive reals

% Statement environments
\theoremstyle{definition}
% % Exercises
\newtheorem{ejer}{Ejercicio} 
\newtheorem*{ejer*}{Ejercicio} % No-number
% % Examples
\newtheorem{ejem}{Ejemplo}
\newtheorem*{ejem*}{Ejemplo} % No-number
% % Definitions
\newtheorem{defn}{Definición}


\title{Proyecto 1\\ ISIS1105 202520}
\author{Rubén Manrique}

\begin{document}
\maketitle

\section*{0. Objetivos}

\section{Condiciones generales}

\section{Descripción del problema}	
En la ciudad futurista de Tecnotown se celebra anualmente el Festival de los
Robots Creativos. Uno de los desafíos consiste en que cada robot debe distribuir
una cantidad total de energía $n$ en exactamente $k$ celdas, de modo que la suma de
las energías asignadas a cada celda corresponda al total de energía $n$.

\subsection{Puntuación}
Una celda otorga puntos de acuerdo a los dígitos de la expansión decimal de la
cantidad de energía almacenada en ella. Sólamente los dígitos múltiplos de 3 otorgan
puntos, de acuerdo a su posición en la expansión decimal. Las únicas posiciones
relevantes son las primeras cinco, leídas de derecha a izquierda. La fórmula del
puntaje entregado por una celda es la siguiente:

\[
	P= v_0P_0 + v_1 P_1 + v_2 P_2 + v_3 P_3 + v_4 P_4
\]

donde $P_i$ es el puntaje asignado por los jueces a la posición $i$-ésima y
\[
	v_i= \alpha(d_i)
\]

donde
\[
	\alpha(m)=
	\begin{cases}
		0,\ \ m \ \mathbf{mod}\ 3 \not= 0\\
		m/3,\ \ m\ \mathbf{mod}\ 3 = 0
	\end{cases}
\]

donde

\[
	E = d_010^0 + d_110^1 + d_210^2 + d_310^3 + d_410^4 + q10^5
\]

y $E$ es la cantidad de energía almacenada en la celda.

\subsection{Ejemplos}
\end{document}
