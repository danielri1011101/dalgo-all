\documentclass{amsart}
\usepackage[utf8]{inputenc}
\usepackage{amsmath}
\usepackage{listings}

% Number sets
\newcommand{\NN}{\mathbf{N}} % Natural numbers
\newcommand{\ZZ}{\mathbf{Z}} % Integers
\newcommand{\Zpos}{\ZZ^{+}} % Positive integers
\newcommand{\RR}{\mathbf{R}} % Real numbers
\newcommand{\Rpos}{\RR^{+}} % Positive reals

% Statement environments
\theoremstyle{definition}
% % Exercises
\newtheorem{ejer}{Ejercicio} 
\newtheorem*{ejer*}{Ejercicio} % No-number
% % Examples
\newtheorem{ejem}{Ejemplo}
\newtheorem*{ejem*}{Ejemplo} % No-number
% % Definitions
\newtheorem{defn}{Definición}


\title{Proyecto 1 \\ ISIS1105 202520}
\author{Rubén Manrique}

\begin{document}
\maketitle

\section*{0. Objetivos}

\section{Condiciones generales}

\section{Descripción del problema}	
En la ciudad futurista de Tecnotown se celebra anualmente el Festival de los
Robots Creativos. Uno de los desafíos consiste en que cada robot debe distribuir
una cantidad total de energía $n$ en exactamente $k$ celdas, de modo que la suma de
las energías asignadas a cada celda corresponda al total de energía $n$.

\bigskip
El jurado del festival considera creativas únicamente aquellas celdas cuyos números
contienen los dígitos 3, 6 o 9. La cantidad de puntos de creatividad obtenida por
una celda depende tanto del dígito como de su posición en el número, según la
siguiente tabla:
\end{document}
