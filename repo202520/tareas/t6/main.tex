\usepackage[utf8]{inputenc}
\usepackage{amsmath}
\usepackage{listings}

% Number sets
\newcommand{\NN}{\mathbf{N}} % Natural numbers
\newcommand{\ZZ}{\mathbf{Z}} % Integers
\newcommand{\Zpos}{\ZZ^{+}} % Positive integers
\newcommand{\RR}{\mathbf{R}} % Real numbers
\newcommand{\Rpos}{\RR^{+}} % Positive reals

% Statement environments
\theoremstyle{definition}
% % Exercises
\newtheorem{ejer}{Ejercicio} 
\newtheorem*{ejer*}{Ejercicio} % No-number
% % Examples
\newtheorem{ejem}{Ejemplo}
\newtheorem*{ejem*}{Ejemplo} % No-number
% % Definitions
\newtheorem{defn}{Definición}


\begin{document}
\maketitle

\section*{ALGORITMOS APROXIMADOS}

\bigskip
\bigskip
\bigskip

%
\section{Cover aproximado}
\subsection{Casos límite} 
\begin{enumerate}
	\item De un ejemplo de un grafo para el cual \texttt{approxCov} retorna un
		cover minimal si se ordenan adecuadamente las aristas.
	\item De un ejemplo de un grafo para el cual \texttt{approxCov} es incapaz
		de retornar un cover minimal sin importar cómo se ordenen las
		aristas.
	\item Sea el \emph{covering number} de un grafo el tamaño de sus covers
		minimales. Sea $q_g$ el covering number del grafo $g$. De una
		condición sobre $q_g$ que garantice que
		\[
			q_g > \# \ \texttt{approxCov(} g \texttt{)}.
		\]
\end{enumerate}

\subsection{Aproximación greedy} De un algoritmo avaro de complejidad temporal
$O(n_g)$ que retorna un cover minimal si $g$ es un árbol. Es decir, si $g$ es
conexo y tiene exactamente $n_g-1$ aristas.

%
\newpage
\section{TSP métrico aproximado}
\subsection{\emph{Closest-point} heuristic} Sea $g$ un grafo métrico. Considere
$\{0\}$ como un ciclo de longitud 1, y agregue sucesivamente el nodo $u$ cuyo vecino
más cercano $v$ se encuentre dentro del ciclo. Modifique las conexiones alrededor
de $v$ si es necesario, y cierre siempre el ciclo. El proceso termina cuando el ciclo
es de longitud $n_g$. Demuestre que este algoritmo es 2-aproximado.

%
\newpage
\section{Set-cover aproximado}
\subsection{Ejecución manual} Considere cada una de las siguientes palabras como un
conjunto de letras:
\[
	\begin{array}{l}
		\mathcal{F}=
	\{
		\texttt{arid, } \texttt{dash, } \texttt{drain, } \texttt{heard, }
		\texttt{lost, }\\
		\ \ \ \ \ \ \ \ \texttt{nose, } \texttt{shun, } \texttt{slate, }
		\texttt{snare, } \texttt{thread}
	\}.
	\end{array}
\]

De este modo, $\mathcal{F}$ es un cover del conjunto
\[
	X=
	\{
		\texttt{a, d, e, h, i, l, m, n, o, r, s, t, u}
	\}.
\]

Determine el retorno de
\(
	\texttt{approxSetCov}(X, \mathcal{F})
\)
cuando en la línea 4 el desempate ocurre por escoger la palabra que aparece primero
en el diccionario.

\subsection{Implementación en tiempo cuadrático}\begin{enumerate}
	\item Implemente (definiendo todas las estructuras de datos manualmente) el
		algoritmo \texttt{appoxSetCov} de modo que su complejidad temporal
		sea
		\[
			O
			\left(
				\sum_{S \in \mathcal{F}} \#\, S
			\right).
		\]

	\item Explique por qué en general
		\[
			O
			\left(
				\sum_{S \in \mathcal{F}} \#\, S
			\right)
			=
			O(n^2)
		\]
		donde $n= \#\, X$.
\end{enumerate}
\end{document}
