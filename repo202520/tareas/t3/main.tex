\documentclass{amsart}
% Packages

% Sets

% Formal statements


\title{ISIS1105\\ Tarea 3\\ 202520 Sección 2}
\author{Daniel R. Barrero R.}

\begin{document}
\maketitle

%

\section{Complejidad} Para los ejercicios 1.1-1.4 consulte el archivo adjunto 
\texttt{Tarea3.java}.

\subsection{} Determine la complejidad temporal y espacial del algoritmo
\texttt{problem00} y justifique su respuesta.

\subsection{} Determine la complejidad temporal y espacial del algoritmo
\texttt{problem01} y justifique su respuesta.

\subsection{} Determine la complejidad temporal y espacial del algoritmo
\texttt{problem10} y justifique su respuesta.

\subsection{} Determine la complejidad temporal y espacial del algoritmo
\texttt{problem11} y justifique su respuesta.

\bigskip
Para los siguientes ejercicios demuestre que $\tau \in O(f)$.

\subsection{} \(\tau(n)= n + \log n,\ f(n)= n\).

\subsection{} \(\tau(n)= 12n,\ f(n)= 2\lfloor n/7 \rfloor + 3\lceil n/8 \rceil\).

\subsection{} \(\tau(n)= 12\lceil n\log n \rceil,\ f(n)= n^2\).


%

\section{Programación dinámica}
Para los ejercicios indicados más adelante, realice lo siguiente.
\begin{enumerate}
	\item Un archivo de Python o Java que resuelva el problema respetando el formato I/O del enunciado. Si su solución no respeta dicho formato, obtendrá 0 puntos. Además, su archivo debe llamarse \texttt{EjercicioX.py} o \texttt{EjercicioX.java} donde \texttt{X} es el número del ejercicio. Por ejemplo, si resuelve \emph{10261 Ferry Loading} y decide hacerlo en Java, su archivo debe llamarse \texttt{Ejercicio10261.java}. Si no respeta este formato, obtendrá 0 puntos. La puntuación será otorgada de acuerdo al número de casos de prueba que su archivo resuelva exitosamente dentro de un tiempo límite pre-establecido. Los casos de prueba que serán usados para calificarles son mucho más difíciles que los ejemplos del enunciado. [\textbf{5 pts}.]
	\item Una descripción verbal de su solución donde explique cómo su archivo resuelve el problema e indique las complejidades espacial y temporal en notación $O$, justificando su elección. [\textbf{5 pts}.]
	\item Un algoritmo avaro en formato libre (pseudocódigo o código fuente) capaz de resolver correctamente algunas instancias del problema, junto con dos ejemplos donde entrega la solución óptima y dos ejemplos donde no. Dichos ejemplos deben ser diferentes a los del enunciado y deben estar acompañados de una justificación verbal de por qué la solución que entregan es o no óptima según sea el caso. [\textbf{5 pts}.]
	\item[]
	\item[] Estos últimos dos componentes del ejercicio deben entregarse por escrito en un archivo de nombre \texttt{EjercicioX.pdf} donde \texttt{X} es el número del ejercicio.
\end{enumerate}

\bigskip
En el archivo \texttt{ejersPD.zip} encuentra dieciséis ejercicios de programación dinámica, de los cuales debe realizar lo indicado arriba para los siguientes tres:
\begin{itemize}
	\item 166 Making change
	\item 437 The tower of Babylon
	\item 562 Dividing coins
\end{itemize}
Los demás ejercicios le sirven de práctica.
\end{document}
