\usepackage[utf8]{inputenc}
\usepackage{amsmath}
\usepackage{listings}

% Number sets
\newcommand{\NN}{\mathbf{N}} % Natural numbers
\newcommand{\ZZ}{\mathbf{Z}} % Integers
\newcommand{\Zpos}{\ZZ^{+}} % Positive integers
\newcommand{\RR}{\mathbf{R}} % Real numbers
\newcommand{\Rpos}{\RR^{+}} % Positive reals

% Statement environments
\theoremstyle{definition}
% % Exercises
\newtheorem{ejer}{Ejercicio} 
\newtheorem*{ejer*}{Ejercicio} % No-number
% % Examples
\newtheorem{ejem}{Ejemplo}
\newtheorem*{ejem*}{Ejemplo} % No-number
% % Definitions
\newtheorem{defn}{Definición}


\begin{document}
\maketitle

\section{Teorema de Cook}
\subsection{Máquinas de Turing} Defina una máquina de Turing
\(
	M_\texttt{pow}= (K, \Gamma, \delta, s, h)
\)

Para decidir el lenguaje
\[
	L_\texttt{pow}= \{\texttt{0}^{2^n}\} \subset \{\texttt{0}\}^*
	.
\]

Use
\(
	\Gamma=
	\{
		\mbox{``\texttt{No}'', \texttt{0, x, }$\sqcup$,\ ``\texttt{Yes}''}
	\}
\)
para resolver este problema.

\subsection{Funciones booleanas} La función de transición de una máquina de Turing
tiene tipo
\[
	\delta : K \times \Gamma \to K \times \Gamma \times
			\{
				\texttt{L, R}
			\}
			.
\]

Sea 
\(
	\mathbf{F}_\delta
\)
su codificación como función entre bit-strings. Tiene tipo 
\[
	\mathbf{F}_\delta :
		\langle
			K \times \Gamma
		\rangle
			\to
		\langle
			K \times \Gamma \times
				\{
					\texttt{L, R}
				\}
		\rangle
		.
\]

\begin{enumerate}
	\item Determine el tamaño de los conjuntos
		\(
			\langle
				K \times \Gamma
			\rangle
		\)
		y
		\(
			\langle
				K \times \Gamma \times
					\{
						\texttt{L, R}
					\}
			\rangle
			.
		\)
	\item Sea
		\(
			F_\delta^i :
				\langle
					K \times \Gamma
				\rangle
					\to
				\{
					0, 1
				\}
		\)
		la $i-$ésima componente de
		\(
			\mathbf{F}_\delta
		\)
		de su solución al ejercicio 1.1. Escoja dos valores de $i$ y
		calcule los circuitos booleanos equivalentes a estas
		\(
			F_\delta^i
		\).
\end{enumerate}

%--

\section{Reducciones}
\subsection{3-CNF-SAT (Incluye uso de IA)}
Lea la demostración de la $NP-$completitud del problema \texttt{3-CNF-SAT}. La puede
encontrar como la demostración al teorema 34.10 del texto guía.

\begin{enumerate}
	\item Explique la demostración en sus propias palabras. Está permitido que pida
		ayuda a personas o IAs, pero está obligado/a a intentarlo por su cuenta
		inicialmente. Responda en el siguiente orden
		\begin{enumerate}
			\item Escriba aproximadamente cuánto tiempo (en minutos) dedicó a
				esta tarea por su cuenta. El gran profesor Benjamin
				C. Pierce de la universidad de Pennsylvania aconseja
				buscar ayuda para un problema después de mínimo 60
				minutos y máximo 180 minutos de trabajo 100\%
				individual.
			\item Si necesitó pedir ayuda, explique con el mayor detalle
				posible cuáles fueron las dificultades que tuvo al
				seguir el argumento lógico, tales como
				\begin{itemize}
					\item Símbolos matemáticos y otros tipos de
						notación difíciles de entender (escriba
						específicamente cuáles).
					\item Conclusiones inesperadas como
						\begin{quote}
							``\ldots \emph{por lo cual la
							solución a la ecuación
							diferencial es}
							\(
								f(x) = \sin(x) + 2.
							\)'' 
						\end{quote}
						o como por ejemplo 
						\begin{quote}
							``\ldots \emph{y vemos que el
								área de la esfera es}
							\(
								4\pi r^2.
							\)'' 
						\end{quote}
						donde a usted le queda un vacío lógico
						para aceptar que se trata de una
						conclusión válida.
				\end{itemize}
			\item Escriba si pidió ayuda a personas, IAs, o ambas. En caso
				de haber usado IAs, diga cuál(es) e incluya tanto sus
				prompts como las respuestas que recibió. Esto último
				lo puede incluir como texto o como captura de pantalla,
				según prefiera.
		\end{enumerate}
\end{enumerate}

\subsection{Reducir a decisión}
\begin{enumerate}
	\item Demuestre que \texttt{Cov} $\leq_P$ \texttt{CovD}.
	\item Demuestre que \texttt{Ham} $\leq_P$ \texttt{HamD}.
	\item Demuestre que \texttt{TSP} $\leq_P$ \texttt{TSPD}.
\end{enumerate}

\subsection{NP-hardness de TSP}
\begin{enumerate}
	\item[] Demuestre que \texttt{HamD} $\leq_P$ \texttt{TSPD}.
\end{enumerate}

%--

\section{LSP y Factorización}
\subsection{} Demuestre que \texttt{LSPD} es $NP-$completo.
\begin{quote}
	\emph{\textbf{Ayuda:} Demuestre que \mbox{\texttt{HamD} $\leq_P$ \texttt{LSPD}}
		para hardness.}
\end{quote}
\end{document}
