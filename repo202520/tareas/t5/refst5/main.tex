\usepackage[utf8]{inputenc}
\usepackage{amsmath}
\usepackage{listings}

% Number sets
\newcommand{\NN}{\mathbf{N}} % Natural numbers
\newcommand{\ZZ}{\mathbf{Z}} % Integers
\newcommand{\Zpos}{\ZZ^{+}} % Positive integers
\newcommand{\RR}{\mathbf{R}} % Real numbers
\newcommand{\Rpos}{\RR^{+}} % Positive reals

% Statement environments
\theoremstyle{definition}
% % Exercises
\newtheorem{ejer}{Ejercicio} 
\newtheorem*{ejer*}{Ejercicio} % No-number
% % Examples
\newtheorem{ejem}{Ejemplo}
\newtheorem*{ejem*}{Ejemplo} % No-number
% % Definitions
\newtheorem{defn}{Definición}


\begin{document}
\maketitle

\section{Funciones de bit-strings}
\begin{dfn}
	Sea 
	\(
		\alpha \in \{0, 1\}^* 
	\)
	un bit-string de longitud $n$. Denotamos su
	$i-$ésimo bit mediante
	\(
		\alpha[i]
	\)
	donde
	\(
		i \in
		\{
			0, 1, \ldots, n-1
		\}
	\).
\end{dfn}

\begin{exl}
	Si
	\(
		\alpha= \texttt{1011}
	\)
	entonces
	\(
		\alpha[1] = \texttt{0}
	\)
	y
	\(
		\alpha[2] = \texttt{1}
	\).
\end{exl}

\begin{dfn}
	Sean \texttt{A} y \texttt{B} conjuntos de bit-strings, y sea
	\[
		\mathbf{F} : \texttt{A} \to \texttt{B}
	\]
	una función. Su $i-$ésima componente es la función
	\[
		F^i : \texttt{A} \to \{0, 1\}
	\]
	dada por
	\[
		F^i(\alpha)= \mathbf{F}(\alpha)[i].
	\]
\end{dfn}

%--

\section{Problemas computacionales}

\begin{dfn}
	Sean $\hat{X}$ y $\hat{Y}$ tipos y sea
	\[
		F : \hat{X} \to \hat{Y}
	\]
	una función. Esta se dice \emph{calculable} si existe un algoritmo
	\[
		\hat{Y}\ \texttt{computeF}(\hat{X}\ x)
	\]
	tal que
	\(
		\texttt{computeF}(x) = F(x)
	\)
	para todo $x \in \hat{X}$.
\end{dfn}

\begin{dfn}
	Sea
	\(
		F : \hat{X} \to \hat{Y}
	\)
	una función calculable. El \emph{problema computacional} asociado a ella
	consiste en escribir el algoritmo \texttt{computeF}.

	\medskip
	\begin{quote}
		 Usamos
			\(
				\llbracket
					F
				\rrbracket
			\)
		para denotar el problema computacional asociado.
	\end{quote}
\end{dfn}

\begin{dfn}
	Sea
	\(
		F : \hat{X} \to \hat{Y}
	\)
	una función calculable y sea
	\(
		Z \subset \hat{Z}
	\)
	un problema de decisión. Decimos que
	\(
		\llbracket
			F
		\rrbracket
	\)
	se \emph{reduce} a
	\(
		Z \subset \hat{Z}
	\)
	si asumir que
	\(
		\tau_{\texttt{checkZ}} \in O(1)
	\)
	permite escribir \texttt{computeF} con complejidad temporal
	\(
		O(n^k)
	\).
\end{dfn}
\end{document}
