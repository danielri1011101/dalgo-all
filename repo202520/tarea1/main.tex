\documentclass{amsart}
\usepackage[utf8]{inputenc}
\usepackage{amsmath}
\usepackage{listings}

% Number sets
\newcommand{\NN}{\mathbf{N}} % Natural numbers
\newcommand{\ZZ}{\mathbf{Z}} % Integers
\newcommand{\Zpos}{\ZZ^{+}} % Positive integers
\newcommand{\RR}{\mathbf{R}} % Real numbers
\newcommand{\Rpos}{\RR^{+}} % Positive reals

% Statement environments
\theoremstyle{definition}
% % Exercises
\newtheorem{ejer}{Ejercicio} 
\newtheorem*{ejer*}{Ejercicio} % No-number
% % Examples
\newtheorem{ejem}{Ejemplo}
\newtheorem*{ejem*}{Ejemplo} % No-number
% % Definitions
\newtheorem{defn}{Definición}

\title{ISIS1105 - 202520 \\ Tarea 1 \\ Sección 2}
\author{Daniel R. Barrero R.}

\begin{document}

\maketitle

\section{Analizar código}

Para los siguientes algoritmos, determine su complejidad espacial y temporal en
notación O-grande.

\begin{ejem*}
	Para el algoritmo de ordenamiento visto en clase.

	\texttt{insertionSort:}
	\begin{itemize}
		\item Complejidad temporal: $O(n^2)$
		\item Complejidad espacial: $O(1)$
	\end{itemize}
\end{ejem*}

\begin{ejer}
	Este es el primer algoritmo.

	\lstinputlisting[language=Java, firstline=8, lastline=14]{T1Code.java}
\end{ejer}

\begin{ejer}
	Este es el segundo algoritmo.

	\lstinputlisting[language=Java, firstline=16, lastline=29]{T1Code.java}
\end{ejer}

\begin{ejer}
	Este es el tercer algoritmo.

	\lstinputlisting[language=Java, firstline=31, lastline=47]{T1Code.java}
\end{ejer}

\begin{ejer}
	Este es el cuarto algoritmo.

	\lstinputlisting[language=Java, firstline=49, lastline=59]{T1Code.java}
\end{ejer}

\section*{$7/4$. Calentamiento}

Estos ejercicios no suman puntos de la tarea, son ejercicios fáciles para
practicar el lenguaje formal.

\begin{ejer*}
	Demuestre que $O(f) = O(g)$ si y sólo si $f \in O(g)$ y $g \in O(f)$.
\end{ejer*}

\section{Complejidad}

\begin{ejer}
	¿Verdadero o falso? Justifique su respuesta con una demostración o un
	contraejemplo, según corresponda.

	\begin{enumerate}
		\item $2^{n+1} \in O(2^n)$
		\item Para cualquier función $f : \NN \to \Rpos$ si $f \in O(n)$
			entonces $f^2 \in O(n^2)$.
		\item Para cualquier función $f : \NN \to \Rpos$ si $f \in O(n)$
			entonces $2^f \in O(2^n)$.
	\end{enumerate}
\end{ejer}

\begin{ejer}
	Sean $f, g : \NN \to \Rpos$. Demuestre que si $\lim_{n \to \infty} f(n)/g(n) \in \Rpos$ entonces $O(f) = O(g)$.
\end{ejer}

\begin{ejer}
	Demuestre que para cualesquiera $a, b > 1$ se cumple que $\log_a n \in O(\log_b n)$.
\end{ejer}

\begin{ejer}
	¿Es cierto que $2^{\log_a n} \in O(2^{\log_b n})$ para cualesquiera $a, b > 1$? Justifique su respuesta con una demostración o un contraejemplo, según corresponda.
\end{ejer}

\newpage
\thispagestyle{plain}

\section*{Definiciones}

Esta es la definición de la notación O-grande (\emph{big-O notation} en inglés).

\begin{defn}
	Sea $f : \NN \to \Rpos$. Entonces

	$$
	O(f) = \{\tau : \NN \to \Rpos \ | \ \exists C \in \Rpos, N \in \NN:
	\ \forall n \geq N: \ \tau(n) \leq C f(n)\}
	$$
\end{defn}

\end{document}
